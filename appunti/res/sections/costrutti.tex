\newpage
\section{Elementi principali del linguaggio}

\textbf{Rust}:

\begin{itemize}
  \item è un linguaggio \textbf{staticamente tipato}
  \item utilizza \textbf{type inference}
  \item è ``\textbf{expression-based}'' (le espressioni ritornano un
valore, gli statements no)
\end{itemize}

\subsection{Costrutti - parti più importanti}

\subsubsection{Variable bindings}

Le variabili vanno \textbf{sempre} inizializzate e legano un tipo a un valore.

``Vivono'' solo all'interno del blocco in cui vengono dichiarate e possono
essere eventualmente oscurate.

Uso di pattern matching:

\begin{lstlisting}
let (x,y) = (1,2)
\end{lstlisting}

Di default le variabili sono \textbf{immutabili} (a meno che non si usi
esplicitamente la keyword \textbf{mut})

\subsubsection{Funzioni}

Nonostante Rust utilizzi type inference, i tipi degli argomenti di una funzione
vanno dichiarati (a scopo documentativo).

Il tipo di ritorno va scritto dopo la freccia -> e deve essere
\textbf{unico}.

\begin{lstlisting}
fn function(x1: t1, ... , xn: tn) -> tr
\end{lstlisting}

L'ultima linea del corpo di una funzione determina cosa ritornare.
Perciò se la funzione ritorna qualcosa l'ultima linea non può essere uno
statement, deve essere una espressione (niente punto e virgola).

Statements: terminano con ; le espressioni no.

\subsubsection{Tipi primitivi}

Booleani, caratteri, numerici, arrays, "slices", stringhe, tuple e funzioni.

Un tipo numerico è solitamente composto da due parti:

\begin{itemize}
\item \textbf{categoria} es. signed, unsigned, integer, float \dots
\item \textbf{dimensione} es 8, 16, 32, 64 bits
\end{itemize}

Altri tipi generici utili: opzione \textbf{Option<T>}, singoletto:
\textbf{Unit ()}, risultato: \textbf{Result<T>}.

Uno slice è un riferimento o una ``vista'' a un'altra struttura dati, ad esempio
un array.

Permette un accesso sicuro ed efficiente a una porzione dell'array senza bisogno
di farne prima una copia.

Sintassi: \& simile ai riferimenti in C/C++, seguita da [ ] per il range (\&[ ]).

\begin{lstlisting}
let a = [0, 1, 2, 3, 4];
// Uno slice con tutti gli elementi di a.
let complete = &a[..];
// Uno slice di a con 1,2 e 3.
let middle = &a[1..4];
\end{lstlisting}

\subsubsection{If}
If è una espressione e il risultato può essere assegnato a una variabile.
If senza else ritorna ().

I rami if ed else devono ritornare tipi compatibili tra loro.

\subsubsection{Cicli}

Sintassi:
\begin{lstlisting}
for var in expression {
    code
}
\end{lstlisting}

\subsubsection{Vettori}

L'indexing va fatto con il tipo usize:

\begin{lstlisting}
let v = vec![1,2,3];
let i: usize = 0;
let j: i32 = 0;
v[i]; // ok
v[j]; // no
\end{lstlisting}

Se si va out-of-scope, la funzione main non ritorna e il thread si ferma
(panica).

Meglio usare \textbf{get o get\_mut}, che in caso di out-of-scope ritornano
\textbf{None}.

Iterazione: per riferimento (mutabile o immutabile), oppure prendendo
l'ownership del vettore (vedi sezione \ref{sec:ownership}). Nel secondo caso
il vettore NON potrà più essere usato.

\begin{lstlisting}
let mut v = vec![1, 2, 3, 4, 5];

for i in &v { // per riferimento immutabile
    println!("A reference to {}", i);
}

for i in &mut v { // per riferimento mutabile
    println!("A mutable reference to {}", i);
}

for i in v { // prende l'ownership
    println!("Take ownership of the vector and its element {}", i);
} // dopo v diventa inutilizzabile
\end{lstlisting}

\subsubsection{Match}

\subsection{Traits e Traits objects}

Un match deve essere completo, cioè deve coprire ogni possibilità.
Se ogni possibilità viene coperta da un match, l’aggiunta di varianti a un
enum causerà un errore di compilazione, invece che un problema durante
l'esecuzione.
Questo tipo di assistenza permette di modificare il codice Rust liberamente.

\subsubsection{Traits}

Si può pensare a un trait come a un'interfaccia: è una notazione per aggiungere
un comportamento extra a un particolare tipo.

Un trait è una caratteristica del linguaggio Rust che comunica al compilatore
che tipo di funzionalità viene fornita da un certo tipo.

Nel seguente esempio viene implementato il trait HasArea per Circle:

\begin{lstlisting}
struct Circle {
    x: f64,
    y: f64,
    radius: f64,
}

trait HasArea {
// self si riferisce a un'istanza del tipo che implementa il trait
    fn area(&self) -> f64; // solo la segnatura della funzione
}

impl HasArea for Circle { // implementazione
    fn area(&self) -> f64 {
        std::f64::consts::PI * (self.radius * self.radius)
    }
}
\end{lstlisting}

\subsubsection{Trait objects}

Rust preferisce il dispatch statico ma fornisce anche il dispatch dinamico,
tramite i trait objects:

\begin{lstlisting}
trait Foo {
    fn method(&self) -> String;
}

// Ora viene implementato per utf8 e String:
impl Foo for u8 {
    fn method(&self) -> String { format!("u8: {}", *self) }
}

impl Foo for String {
    fn method(&self) -> String { format!("string: {}", *self) } }
\end{lstlisting}

Possiamo usare questo trait per il dispatch statico:

\begin{lstlisting}
fn do_something<T: Foo>(x: T) {
    x.method();
}

fn main() {
    let x = 5u8;
    let y = "Hello".to_string();
    do_something(x);
    do_something(y);
}
\end{lstlisting}

In questo esempio Rust usa la ``\textbf{monomorphization}'' per il dispatch
statico. Ciò significa che verranno create delle versioni specializzate di
do\_something per u8 e per String.

Le chiamate a do\_something verranno rimpiazzate con queste funzioni
specializzate.

Rust genera quindi qualcosa di simile:

\begin{lstlisting}
fn do_something_u8(x: u8) {
    x.method();
}

fn do_something_string(x: String) {
    x.method();
}

fn main() {
    let x = 5u8;
    let y = "Hello".to_string();

    do_something_u8(x);
    do_something_string(y);
}
\end{lstlisting}

Il dispatch statico permette l'inlining delle chiamate di funzioni (sostituzione
della chiamata con il corpo della funzione) perché il chiamante è noto a tempo
di compilazione.

L'inlining è una buona ottimizzazione che però richiede un tradeoff, cioè il
``code bloat'' (rigonfiamento di codice).

Rust fornisce il dispatch dinamico attraverso i trait objects, come \&Foo, cioè
oggetti che contengono un valore di un tipo qualsiasi che implementi il trait
dato.

Il tipo preciso può essere conosciuto solo a runtime.

Dispatch dinamico con trait objects e casting:

\begin{lstlisting}
fn do_something(x: &Foo) {
    x.method();
}

fn main() {
    let x = 5u8;
    do_something(&x as &Foo);
}
\end{lstlisting}

Con coercizione:

\begin{lstlisting}
fn do_something(x: &Foo) {
    x.method();
}

fn main() {
    let x = "Hello".to_string();
    do_something(&x);
}
\end{lstlisting}

Do\_something in questi due casi di dispatch dinamico \textbf{non} viene
specializzata.

\textbf{Svantaggi}: virtual function calls più lente e no inlining.

\subsubsection{Object safety}

Non tutti i trait possono essere usati per formare trait objects.
Ad esempio, i vector implementano il trait Clone e se proviamo a creare un trait
object:

\begin{lstlisting}
let v = vec![1, 2, 3];
let o = &v as &Clone;
\end{lstlisting}

viene segnalato l'errore:

\begin{lstlisting}
error: cannot convert to a trait object because trait core::clone::Clone is
not object-safe [E0038]

let o = &v as &Clone;
        ^~
note: the trait cannot require that Self : Sized

let o = &v as &Clone;
        ^~
\end{lstlisting}

L'errore dice che clone non è \textbf{object-safe}.

\begin{itemize}
  \item Un \textbf{trait} è object safe se non richiede self: sized e tutti i suoi
metodi sono object safe.

  \item Un \textbf{metodo} è object-safe se richiede self: sized oppure...
\begin{itemize}
  \item Non ha type parameters
  \item Non usa self
\end{itemize}
\end{itemize}
