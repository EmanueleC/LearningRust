\newpage
\section{Lifetimes, Mutability, Closures}

Un problema comune è quello dei \textbf{dangling pointers} o
\textbf{use after free}.
Si verifica ad esempio quando un riferimento vive più a lungo della risorsa
cui si riferisce.

Set di operazioni che portano al problema:

\begin{enumerate}
\item Gestisco un qualche tipo di risorsa;
\item Passo a qualcun altro un riferimento alla risorsa;
\item Decido che ho finito di usare la risorsa e la dealloco, mentre qualcun
altro ha un riferimento a essa;
\item L’altra persona decide di usare il riferimento.
\end{enumerate}

Il passo 4 non deve mai verificarsi prima del passo 3. Si consideri il seguente
esempio:

\begin{lstlisting}
fn skip_prefix(line: &str, prefix: &str) -> &str {
    // ...
}

let line = "lang:en=Hello World!";
let lang = "en";

let v;
{
    let p = format!("lang:{}=", lang);  // -+ `p` entra nello scope.
    v = skip_prefix(line, p.as_str());  //  |
}                                       // -+ `p` esce dallo scope.
println!("{}", v);
\end{lstlisting}

In questo caso la safety di println! dipende da skip\_prefix, che potrebbe
ritornare line, ancora nello scope, oppure p che invece è andata fuori.

\subsection{Visibility}

Il codice dell’esempio precedente non viene compilato. Occorre informare il
compilatore del tempo di vita dei riferimenti della funzione skip\_prefix:

\begin{lstlisting}
fn skip_prefix<'a, 'b>(line: &'a str, prefix: &'b str) -> &'a str {
    // ...
}
\end{lstlisting}

La sintassi 'a, 'b indica il tempo di vita dei riferimenti.
In questo caso si ritorna il riferimento con tempo di vita 'a (il primo),
cioè line.

Println! è sicura e il compilatore non segnala nulla.

Le \textbf{annotazioni di lifetimes} (tempo di vita) sono descrittive e non
prescrittive. Significa che la validità di un riferimento è determinata dal
codice e non da tali annotazioni.

\subsection{Mutability}

Si può introdurre la mutabilità in Rust con la keyword \textbf{mut}, sia per i
bindings sia per i riferimenti.

\textbf{Non è possibile avere una struct con campi immutabili e anche campi
mutabili}.

\subsection{Generics}

A volte c’è la necessità di scrivere una funzione che accetti argomenti di
diversi tipi. Rust permette di raggiungere questo scopo con i generics.
I generics sono chiamati \textbf{polimorfismo parametrico} in teoria dei tipi.

Esempio:

\begin{lstlisting}
enum Option<T> {
    Some(T),
    None,
}
\end{lstlisting}

Option<T> è un generic, dove T è un tipo qualsiasi (ad esempio Option<i32>).
Esempio di generic su due tipi:

\begin{lstlisting}
enum Result<T, E> {
    Ok(T),
    Err(E),
}
\end{lstlisting}

Esempio di funzione + generics:

\begin{lstlisting}
fn takes_anything<T>(x: T) {
    // Operazioni su x.
}
\end{lstlisting}

\subsection{Drop}

È un trait particolare della libreria std di Rust. Permette di eseguire del
codice quando un valore di un tipo che implementa Drop va out of scope.
I valori vanno out of scope in ordine inverso rispetto alla
dichiarazione.
Esempio:

\begin{lstlisting}
struct HasDrop;

impl Drop for HasDrop {
    fn drop(&mut self) {
        println!("Dropping!");
    }
}

fn main() {
    let x = HasDrop;

    // Altre operazioni...

} // x va out of scope
\end{lstlisting}

Quando la x va out of scope il codice di Drop::drop() (unica funzione del trait)
viene eseguito e stampa \textit{Dropping!}.

Drop viene usato per ripulire ogni risorsa associata a una
struttura.

\subsection{Closures}

Si possono vedere come dei wrappers per le funzioni.

Esempio di closure

\begin{lstlisting}
let plus_one = |x: i32| x + 1; // closure
assert_eq!(2, plus_one(1));
\end{lstlisting}

Viene creato un binding (plus\_one) a cui si assegna una closure.
La x è l’argomento e l’espressione x + 1 è il corpo.

Altro esempio:

\begin{lstlisting}
let plus_two = |x| { // multi-line closure
    let mut result: i32 = x;
    result += 1;
    result += 1;
    result
};
\end{lstlisting}

Non occorre scrivere il tipo degli argomenti e la sintassi è leggermente diversa
rispetto alle funzioni.

Le closures possono usare bindings definiti nel loro blocco e anche local
bindings.

\begin{lstlisting}
let num = 5;
let plus_num = |x: i32| x + num;
\end{lstlisting}

Il seguente esempio dà errore:

\begin{lstlisting}
let mut num = 5;
let plus_num = |x: i32| x + num;

let y = &mut num;
error: cannot borrow num as mutable because it is also borrowed as immutable
    let y = &mut num;
\end{lstlisting}

Non è possibile prendere un riferimento mutabile a num se la closure plus\_num
ne ha già preso uno immutabile.

Il problema si risolve limitando la visibilità della closure.

\subsubsection{Move closure}

Qui num viene incrementato di 5:

\begin{lstlisting}
let mut num = 5;
{
    let mut add_num = |x: i32| num += x;
    add_num(5);
}
assert_eq!(10, num);

// qui num rimane 5, grazie alla keyword move

let mut num = 5;
{
    let mut add_num = move |x: i32| num += x;
    add_num(5);
}
assert_eq!(5, num);
\end{lstlisting}

num è un i32 e implementa il trait Copy.
Non viene preso un riferimento mutabile a num ma viene presa l'ownership di una
sua copia.

\subsection{Gestione degli errori}

Rust divide gli errori software in due tipi di errori:

\begin{itemize}
  \item \textbf{Recuperabili}: il problema viene riportato all'utente e si ritenta
l'operazione che ha portato all'errore.
  \item \textbf{Irrecuperabili}: bugs, ad esempio errore di accesso a un array.
\end{itemize}

Rust non permette di lanciare eccezioni (sono assenti), ma permette di usare il tipo
Result<T, E> per errori recuperabili e la macro panic! per interrompere l'esecuzione
nel caso di errori irrecuperabili.

Il tipo Result ha due varianti, Ok ed Err:

\begin{lstlisting}
enum Result<T, E> {
    Ok(T),
    Err(E),
}
\end{lstlisting}

T rappresenta un tipo di valore ritornato solo in caso di successo, mentre
E rappresenta il tipo d'errore.

Esempio di funzione che ritorna un Result (apertura di un file, File::open):

\begin{lstlisting}
use std::fs::File;

fn main() {
    let f = File::open("hello.txt");

    let f = match f {
        Ok(file) => file,
        Err(error) => {
            panic!("There was a problem opening the file: {:?}", error)
        },
    };
}
\end{lstlisting}

Occorre intraprendere diverse azioni a seconda del risultato della funzione.

Se il risultato è Ok, assegnamo il file dentro la variante Ok alla variabile f
così da poterla usare in seguito per operazioni di lettura/scrittura.

Se si è verificato un errore, il valore viene estratto dalla variante Err e
stampato insieme a un messaggio quando il main thread panica.

Nel codice precedente, non si tiene in considerazione il tipo di errore, si
fa la stessa cosa per tutti i casi in cui File::open ritorna Err.

Si può effettuare un match su più errori, come nel codice seguente, che
distingue l'errore NotFound (file non trovato) da tutti gli altri.

Se l'errore è NotFound, si cerca di creare il file con la funzione File::open,
che a sua volta ritorna un Result ed è quindi necessario gestirlo.

\begin{lstlisting}
use std::fs::File;
use std::io::ErrorKind;

fn main() {
    let f = File::open("hello.txt");

    let f = match f {
        Ok(file) => file,
        Err(ref error) if error.kind() == ErrorKind::NotFound => {
            match File::create("hello.txt") {
                Ok(fc) => fc,
                Err(e) => {
                    panic!(
                        "Tried to create file but there was a problem: {:?}",
                        e
                    )
                },
            }
        },
        Err(error) => {
            panic!(
                "There was a problem opening the file: {:?}",
                error
            )
        },
    };
}
\end{lstlisting}

La condizione \texttt{if error.kind() == ErrorKind::NotFound} è chiamata
guardia del match. Si tratta di una condizione extra che rifinisce ulteriormente
il ramo Err del match iniziale.

Il metodo \textbf{unwrap} è un metodo shortcut che fa esattamente lo stesso lavoro
della prima versione di gestione d'errore scritta per File::open(): se il valore
di Result è Ok, unwrap ritorna il valore all'interno di Ok. Se è Err, unwrap richiama
la macro panic!.

\begin{lstlisting}
use std::fs::File;

fn main() {
    let f = File::open("hello.txt").unwrap();
}
\end{lstlisting}
