\newpage
\section{Ownership System}
\label{sec:ownership}

Rust è un linguaggio che tiene particolarmente alla cosiddetta
\textbf{memory safety}, una proprietà che assicura la protezione da bugs e
vulnerabilità quando si ha a che fare con la memoria (errori di accesso,
variabili non inizializzate, memory leaks, \dots).

Per far fronte a queste problematiche, utilizza un sistema chiamato
\textbf{ownership system}, che ha lo scopo di segnalare al programmatore un
utilizzo non sicuro della memoria a tempo di compilazione.

Utilizza forme di astrazione poco costose per raggiungere tali obiettivi
(``\textbf{zero cost abstractions}'').

Il metodo di ownership viene controllato a tempo di compilazione, così non si
hanno costi aggiuntivi a tempo di esecuzione.

Uno svantaggio risiede nei tempi di apprendimento di questi concetti da parte
dei programmatori.

\paragraph{In generale:} un valore possiede sempre e solo un owner alla
volta, che può essere una variabile, una funzione, un thread, \dots

Come viene allocata la memoria per i vettori:

\begin{lstlisting}
fn foo() {
    let v = vec![1, 2, 3];
}
\end{lstlisting}

\begin{itemize}
\item Quando v acquisisce visibilità, un nuovo vettore viene creato sullo stack
e viene allocato dello spazio sullo heap per i suoi elementi
\item Quando v non è più visibile, Rust pulisce tutto ciò che era relazionato
al vettore, anche la memoria sullo heap.
Questo avviene in modo deterministico alla fine dello scope.
\end{itemize}

\subsection{La semantica Move}

Riassumendo: quando l'owner di un valore cambia, non è più possibile usare
l'owner precedente per accedere a quel valore.

\paragraph{Nota:} i prossimi due esempi funzionano solo perché si usa
il tipo vector.

Con i tipi primitivi è diverso, perché implementano il trait copy.

\begin{lstlisting}
let v = vec![1, 2, 3];
let v2 = v; // cambia l'owner

println!("v[0] is: {}", v[0]); // errore, non posso usare v!
println!("v2[0] is: {}", v2[0]); // ok
\end{lstlisting}

Lo stesso accade quando si tenta di utilizzare qualcosa dopo averlo passato come
argomento a una funzione.

L'errore in entrambi i casi è \textbf{use of moved value}, cioè si tenta di
usare un oggetto la cui ownership è stata acquisita da altri.

\begin{lstlisting}
let v = vec![1, 2, 3];
let mut v2 = v;
v2.truncate(2);

// e poi...

v[2] // Rust vieta di usare v (moved value)
\end{lstlisting}

Il tipo \textbf{COPY} - copia, è un trait che \textbf{cambia la regola
sull'ownership}.

\begin{lstlisting}
let v = 1;
let v2 = v;
println!("v is: {}", v); // OK!
\end{lstlisting}

Qui la cosa funziona perché v ha tipo i32 che implementa il trait COPY.

Quando si assegna v a v2 viene fatta una copia completa di v perché il tipo
i32 non richiede puntatori a dati addizionali.

Tutti i tipi primitivi implementano il trait copy.
\textbf{(Vector NON è un tipo primitivo)}

Per restituire l'ownership nelle funzioni:

\begin{lstlisting}
fn foo(v: Vec<i32>) -> Vec<i32> {
    // Operazioni su v.

    // Restituisce l'ownership.
    v
}
\end{lstlisting}

La cosa può però diventare scomoda:

\begin{lstlisting}
fn foo(v1: Vec<i32>, v2: Vec<i32>) -> (Vec<i32>, Vec<i32>, i32) {
    // Operazioni con v1 e v2.

    // Restituisce l'ownership e il risultato della funzione.
    (v1, v2, 42)
}
\end{lstlisting}

\subsection{Borrowing}

Borrowing significa prendere in prestito. Esempio di borrowing:

\begin{lstlisting}
fn main() {

    fn sum_vec(v: &Vec<i32>) -> i32 {
        return v.iter().fold(0, |a, &b| a + b);
    }

    // Borrowing di due vettori e calcolo della loro somma.
    fn foo(v1: &Vec<i32>, v2: &Vec<i32>) -> i32 {
        let s1 = sum_vec(v1);
        let s2 = sum_vec(v2);
        s1 + s2
    }

    let v1 = vec![1, 2, 3];
    let v2 = vec![4, 5, 6];
    let answer = foo(&v1, &v2);
    println!("{}", answer);
}
\end{lstlisting}

Anziché prendere i vettori come argomenti, foo accetta dei riferimenti al tipo
Vec<i32>.
\textbf{ \&T è un tipo riferimento immutabile}, non prende pieno possesso della
risorsa, ma ne \textbf{prende in prestito l'ownership}.

Un binding che prende in prestito qualcosa non dealloca la risorsa quando va
out-of-scope. Ciò significa che al termine di foo è possibile usare ancora i
bindings originali.

C’è un altro tipo di riferimenti: quelli mutabili (\&mut T).
Per accedere al contenuto di un riferimento, si usa * (come in C++).
Il seguente esempio mostra come si può cambiare il contenuto di un riferimento
mutabile:

\begin{lstlisting}
let mut x = 5;
{ // nota: metto un extra scope, altrimenti non funziona
    let y = &mut x;
    *y += 1;
}
println!("{}", x); // stampa 6
\end{lstlisting}

Ci sono delle regole da seguire...

\subsection{Regole per il borrowing in Rust}

Prima di tutto, ogni borrow deve avere uno scope \textbf{non più grande} di
quello dell’owner (non può vivere più a lungo).

Inoltre, possono valere le seguenti condizioni \textbf{in mutua esclusione}:

\begin{itemize}
\item \textbf{Uno o più riferimenti immutabili} a una risorsa
\item Esattamente \textbf{un riferimento mutabile} a una risorsa
\end{itemize}

Un esempio...

\begin{lstlisting}
fn main() {
    let mut x = 5;
    let y = &mut x;
    *y += 1;
    println!("{}", x); // no
}
\end{lstlisting}

Sbagliato perché esiste un riferimento mutabile y, non è possibile usare anche
un riferimento immutabile in println. Il compilatore segnala:

\begin{lstlisting}
error: cannot borrow `x` as immutable because it is also borrowed as mutable
    println!("{}", x);
\end{lstlisting}

Si corregge come segue...

\begin{lstlisting}
let mut x = 5;
{
    let y = &mut x; // -+ borrowing mutabile &mut inizia qui
    *y += 1;        //  |
}                   // -+ ... e finisce qui
println!("{}", x);  // ok
\end{lstlisting}

I riferimenti non devono vivere più a lungo della risorsa a cui si riferiscono:

\begin{lstlisting}
let y: &i32;
{
    let x = 5;
    y = &x;
}
println!("{}", y); //errore
\end{lstlisting}

Il riferimento y è valido solo nello scope in cui vive x, non al di fuori
(analogo ai dangling pointers in C/C++).

Un altro esempio di errore meno evidente:

\begin{lstlisting}
let y: &i32;
let x = 5;
y = &x;
println!("{}", y);
\end{lstlisting}

y è dichiarata per prima e poiché nello stesso scope le risorse sono deallocate
in ordine inverso (prima x e poi y) quindi y vivrà più a lungo di x.
Questo non è ammesso perché y è un riferimento a x.
