\newpage
\section{Message Passing}

Un approccio alla concorrenza che sembra crescere in popolarità è il
\textbf{message passing}, dove i threads (o attori) comunicano inviando fra loro
messaggi che contengono dati.
A tal proposito, lo slogan di Go dice:

``Do not communicate by sharing memory; instead, share memory by
communicating.''

-Effective Go

\subsection{Channels e ownership}

Nel seguente esempio si tenta di usare un valore nel thread dopo averlo inviato
sul canale:

\begin{lstlisting}

use std::thread;
use std::sync::mpsc;

fn main() {
    let (tx, rx) = mpsc::channel();

    thread::spawn(move || {
        let val = String::from("hi");
        tx.send(val).unwrap();
        println!("val is {}", val);
    });

    let received = rx.recv().unwrap();
    println!("Got: {}", received);
}
\end{lstlisting}

Questa è una cattiva idea: una volta che abbiamo inviato il valore a un altro
thread, quel thread può modificarlo o chiamare drop prima che il sender
lo utilizzi di nuovo. Questo potrebbe causare errori dovuti alla non-esistenza
o inconsistenza dei dati.

Se proviamo a compilare l'esempio, Rust segnala:


\begin{lstlisting}

error[E0382]: use of moved value: val
  --> src/main.rs:10:31
   |
9  |         tx.send(val).unwrap();
   |                 --- value moved here
10 |         println!("val is {}", val);
   |                               ^^^ value used here after move
   |
   = note: move occurs because val has type
   std::string::String, which does
   not implement the Copy trait

\end{lstlisting}

Questo errore di concorrenza ha causato un errore di compilazione.
Send prende l'ownership dei suoi parametri e sposta (moves) il valore
in modo da passare l'ownerhip al ricevente.

Questo significa che non è possibile utilizzare il valore di nuovo dopo averlo
inviato; il sistema di ownership controlla che tutto sia ok.

In questo contesto, message passing è molto simile a single ownership: se solo
un thread alla volta può usare della memoria, non c'è il rischio di data race.
